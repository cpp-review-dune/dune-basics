\input{cmake-preamble}

\begin{document}

\frame[plain,noframenumbering]{
	\titlepage
}

\begin{frame}[fragile]
	\frametitle{\href{https://cmake.org}{CMake} ($2000$ - actualidad)}
	Es un \emph{generador de sistemas de compilación} de código
	abierto multiplataforma e independiente del compilador, es decir,
	genera instrucciones para otros sistemas de compilación como
	\href{https://www.gnu.org/software/make}{GNU Make},
	\href{https://ninja-build.org}{Ninja}
	y archivos de proyecto para IDEs populares como
	\href{https://visualstudio.microsoft.com}{Microsoft Visual Studio},
	\href{https://eclipseide.org}{Eclipse},
	\href{https://www.qt.io/product/development-tools}{Qt Creator},
	\href{https://www.jetbrains.com/clion}{CLion},
	\href{https://www.codeblocks.org}{Code::Blocks},
	\href{https://www.sublimetext.com}{Sublime Text},
	\href{https://kate-editor.org}{Kate},
	\href{https://developer.android.com/studio}{Android Studio} y
	\href{https://developer.apple.com/xcode}{Xcode}.
	Se mantiene centrado en soportar compiladores modernos y cadena
	de herramientas.
	Tiene soporte nativo de pruebas de software, empaquetamiento e
	instalación como una parte inherente del proceso de construcción.

	\begin{block}{Resultados del aprendizaje}
		\begin{itemize}
			\item
			
			Construir diversos ejemplos de proyectos de CMake en los
			lenguajes C/C++/Fortran/Python que construyan ejecutables
			u objetos compartidos/estáticos/interfaces.
			
			\item
			
			Correr pruebas con \lstinline{ctest}, \lstinline{catch2},
			\lstinline{gtest} y \lstinline{pytest}.
			
			\item
			
			Usar dependencias de terceros en un proyecto CMake.
		\end{itemize}
	\end{block}

	\begin{block}{Prerrequisitos}
		\begin{itemize}
			\item

			Está orientado a cualquier estudiante que quiera aprender
			a usar efectivamente CMake en un proyecto basado en Dune
			sobre Arch Linux.
			
			\item

			No se asume que esté familiarizado con GNU/Linux o C++,
			pero sí que esté familiarizado con algún lenguaje de
			programación.
			
			\item

			Los ejemplos y ejercicios han sido probados en Arch
			Linux.
		\end{itemize}
	\end{block}
	% \lstinline{cmake} es una herramienta para automatizar la
	% generación de \lstinline{Makefile}.
	% TODO: Teoría
	% Explicar la necesidad de cmake (ver Scott, Rafal)
	% Instalación de cmake en Arch Linux.
	% Mencionar el manual pages de cmake
	% conceptos: build tree, source tree
	% Etapas de cmake con un diagrama.
	% TODO: Práctica
	% Ejemplo: un ejemplo de álgebra lineal con BLAS del libro de Simon.
	% Ejemplo: documentación con doxygen (Berner, Rafal).
	% Ejemplo con ctest: https://coderefinery.github.io/cmake-workshop/testing
\end{frame}

\begin{frame}[fragile]
	\frametitle{Instalación}
	Esta tecnología imprescindible se encuentra disponible en la
	mayoría de repositorios de distribuciones GNU/Linux importantes.
	Si desea instalar una versión actual, lo puede encontrar en el
	repositorio \href{https://archlinux.org/packages/extra/x86_64/cmake}{\lstinline{[extra]}}

	\begin{verbatim}
		[user@host somedir]$ sudo pacman -Syu # Actualizar
		[user@host somedir]$ sudo pacman -S cmake clang graphviz
	\end{verbatim}
\end{frame}

%https://archlinux.org/packages/extra/x86_64/cmake
%https://archlinux.org/packages/extra/x86_64/clang
%https://archlinux.org/packages/extra/x86_64/cppcheck
%https://archlinux.org/packages/extra/x86_64/doxygen
%https://archlinux.org/packages/extra/x86_64/graphviz
%https://archlinux.org/packages/core/any/base-devel
%https://archlinux.org/packages/extra/any/lcov
%https://archlinux.org/packages/extra/any/plantuml
%https://archlinux.org/packages/extra/any/gcovr
%https://archlinux.org/packages/extra/any/python-cpplint
%https://archlinux.org/packages/extra/x86_64/ccache
%https://archlinux.org/packages/extra/x86_64/libpqxx
%https://archlinux.org/packages/extra/x86_64/protobuf
%https://archlinux.org/packages/extra/x86_64/tree
%https://archlinux.org/packages/extra/x86_64/valgrind

\begin{frame}[fragile]
	\frametitle{Un archivo \lstinline{CMakeLists.txt} minimal}
	% Supongamos que tenemos el siguiente script \lstinline{CMakeLists.txt}
	\inputminted{cmake}{CMakeList.txt.sample}

	Entendamos, este script en más detalle.
	En este script definimos los requisitos para la construcción, desde
	el código fuente y objetivos, pasando por las pruebas, empaquetamiento, etc.
	Y delegaremos la tarea de compilación al programa \lstinline{make}.

	\begin{itemize}
		\item

		\href{https://cmake.org/cmake/help/latest/command/cmake_minimum_required.html}{\lstinline{cmake_minimum_required()}}

		indica la versión mínima de \lstinline{cmake} que requiere para ejecutar.

		\item
		
		\href{https://cmake.org/cmake/help/latest/command/project.html}{\lstinline{project()}}

		define el nombre del proyecto, su número de versión y que
		está escrito en el lenguaje de programación C++.
		
		\item

		\href{https://cmake.org/cmake/help/latest/command/set.html}{\lstinline{set()}}

		asigna la variable de entorno, en este caso establece la versión del estándar C++ 20.


		\item

		\href{https://cmake.org/cmake/help/latest/command/message.html}{\lstinline{message()}}
		
		.

		\item

		\href{https://cmake.org/cmake/help/latest/command/add_subdirectory.html}{\lstinline{add_subdirectory()}}

		.

		\item

		\href{https://cmake.org/cmake/help/latest/command/add_executable.html}{\lstinline{add_executable()}}

		crea un ejecutable a partir de un script en C++.
	\end{itemize}
\end{frame}

\begin{frame}
	\frametitle{Primeros pasos con CMake}
	Esperemamos automatizar la construcción escribiendo un script que vaya a través
	de nuestro árbol del proyecto que compile todo.

	Este escript debe evitar compilaciones innecesarias y detecte cuando el script ha sido modificado
	desde la última vez que corrimos.

	También debe conocer cómo enlazar todos los archivos compilados en un binario.

	CMake no puede construir por sí mismo, este delega la tarea a otras herramientas
	en el sistema para desarrollar la compilación actual, enlace y otras tareas.

	Puede pensar como el orquestador del proceso de construcción
	(sabe qué pasos necesita realizarse, cuál es el objetivo final y cómo encontrar a los
	trabajadores correctos y materiales para el trabajo)
	
	\begin{description}
		\item[árbol de construcción (build tree)]

		es la ruta al directorio objetivo / salida.

		\item[árbol de fuente (source tree)]

		es la ruta en el que se encuentra su código fuente.
	\end{description}
\end{frame}

\begin{frame}
	frametitle{Las tres etapas en CMake}
	
	\begin{block}{Etapa de configuración}
		\begin{itemize}
			\item

			Lee los detalles del proyecto guardados en el árbol de construcción
			y preparará un árbol de construcción para la etapa de generación.
			
			\item

			Crea un árbol de construcción vacío y colecciona los detalles del entorno
			en el que se trabaja (arquitectura, compiladores disponibles, enlazadores,
			revisa un problema de prueba simple que puede compilar correctamente.
			archivos)
			
			
						\item

El archivo de configuración del proyecto es analizado y ejecutado (en lenguaje CMake).
Le dice acerca de la estructura del proyecto, objetivos, dependencias (bibliotecas y paquetes CMake).
			
						\item
			
						Guarda información recolectada en el árbol de construcción
						como detalles del sistema, configuración del proyecto, registros,
						archivos temporales; que serán usados en la siguiente etapa.

						\item

						El archivo \lstinline{CMakeCache.txt} es creador para guardar
						variables estables (rutas del compilador y otras herramientas)
						y ahoora tiempo durante la siguiente configuración.
					\end{itemize}
	\end{block}
\end{frame}

\begin{frame}
	\begin{block}{Etapa de generación}
		\begin{itemize}
			\item

			Generará un sistema de construcción para el entorno
			exacto en el que trabaja (\lstinline{Makefile} para GNU Make, \lstinline{Ninja} o IDE).

			\item

			Puede aplicar toques finales de la configuración evaluando
			expresiones generadoras.
		\end{itemize}
	\end{block}

	\begin{block}{Etapa de construcción}
		\begin{itemize}
			\item

			Corriendo las herramientas de construcción apropiadas
			obtendremos los artefactos.

			\item

			Estas heramientas ejecutarán pasos para producir objetivos
			con compiladores, enlazadores, herramientas de análisis
			estático, frameworks de prueba, herramientas de reporte,
			y cualquier otra cosa que pueda pensar.
		\end{itemize}
	\end{block}
\end{frame}

\begin{frame}[fragile]
	\frametitle{Aprendiendo la línea de comando}
	\begin{block}{Herramientas de línea de comando}
		\begin{itemize}
			\item
			
			\href{https://man.archlinux.org/man/cmake.1}{\lstinline{/usr/bin/cmake}}

			ejecutable principal que configura, genera y construye proyectos.
			
			\item
			
			\href{https://man.archlinux.org/man/ctest.1}{\lstinline{/usr/bin/ctest}}
			
			programa de manejador de pruebas usado para correr y
			reportar resultados de pruebas.

			\item
			
			\href{https://man.archlinux.org/man/cpack.1}{\lstinline{/usr/bin/cpack}}

			este es el programa de empaquetamiento usado para generar
			el instalador y paquete de origen.
		\end{itemize}
	\end{block}
		
	\begin{block}{Herramientas interactivas}
		\begin{itemize}
			\item

			\href{https://man.archlinux.org/man/cmake-gui.1}{\lstinline{/usr/bin/cmake-gui}}

			envoltorio gráfico alrededor de \lstinline{cmake}.
			
			\item
			
			\href{https://man.archlinux.org/man/ccmake.1}{\lstinline{/usr/bin/ccmake}}

			envoltorio en modo texto
			(\href{https://es.wikipedia.org/wiki/Curses}{curses}) alrededor de \lstinline{cmake}.
		\end{itemize}
	\end{block}
\end{frame}

\begin{frame}[fragile]
	\frametitle{\lstinline{/usr/bin/cmake}}
	Este binario provee algunos modos de operaciones

	\begin{block}{Generación de un sistema de construcción del proyecto}
		Este es el primer paso requerido para construir nuestro
		proyecto.
	\end{block}
	
	\begin{block}{Construcción de un proyecto}
		.
	\end{block}
\end{frame}

\begin{frame}[fragile]
	\frametitle{Tipos de construcción CMake}

	\begin{itemize}
		\item
		
		Debug

		\item
		
		Release

		\item
		
		RelWithDebInfo

		\item
		
		MinSizeRel

		\item
		
		None
	\end{itemize}
\end{frame}

\begin{frame}[fragile]
	\frametitle{Generación automática de documentación con CMake}
	La documentación es esencial en cualquier proyecto exitoso.

	Veamos cómo integrar Doxygen con CMake para generar
	automáticamante la documentación para proyectos con CMake.

	CMake generará un archivo \lstinline{Doxyfile}.

	Esperamos que Doxygen genere la documentación de la interfaz de
	programación de aplicaciones (API) para cada clase y sus
	diagramas de herencia con \lstinline{dot} de graphviz.
\end{frame}

\begin{frame}[fragile]
	\frametitle{Pruebas con ctest}
	.
\end{frame}

% Reproducible builds, superbuilds

% TODO:
% Se va a separar una diapositiva de dune-cmake cuando esté terminado.
% Ejemplo: que sea de dune con dune-common, dune-geometry cmake.

\begin{frame}[fragile]
	\frametitle{El comando \href{https://gitlab.dune-project.org/core/dune-common/-/raw/master/bin/duneproject}{\lstinline{duneproject}}}

	Es un script asistente escrito en el lenguaje \lstinline{bash}
	que se encuentra en \lstinline{/usr/bin/duneproject}
	dentro del paquete \lstinline{dune-common}.
\end{frame}

\begin{frame}\transblindsvertical
	\frametitle{Referencias}
	%------------------------------------------------------------ 1
	\only<1>{
		\begin{itemize}
			\item Libros
			      \nocite{*}
			      \printbibliography[heading=none,keyword=book]
		\end{itemize}
	}
	%------------------------------------------------------------ 2
	\only<2>{
		\begin{itemize}
			\item Artículos
			      \printbibliography[heading=none,keyword=paper]
		\end{itemize}
	}
	%------------------------------------------------------------ 3
	\only<3>{
		\begin{itemize}
			\item Sitios web
			      \printbibliography[heading=none,keyword=online]
		\end{itemize}
	}
\end{frame}

\end{document}
%https://cliutils.gitlab.io/modern-cmake
%https://www.dune-project.org/sphinx/content/sphinx/core/index.html
%https://www.dune-project.org/sphinx/content/sphinx/core-2.7
%https://github.com/toeb/moderncmake/blob/master/Modern%20CMake.pdf