\input{dune-doxygen-preamble}

\begin{document}

\frame[plain,noframenumbering]{\titlepage}

\begin{frame}
	\frametitle{Objetivos de esta introdución}
	A number of advanced.
	\begin{itemize}
		\item Correr los \href{https://github.com/orgs/cpp-review-dune/packages}{imágenes prediseñadas} con el programa \lstinline{docker} en sistemas tipo Unix.
		\item Crear una malla con \lstinline{gmsh} a través de la API (C/C++, Python, Julia).
		\item Realizar los ejemplos resueltos del \textsc{Dune book}.
	\end{itemize}
\end{frame}

\begin{frame}
	\frametitle{Ideas para explicar Doxygen}
	\begin{description}
	\item Utilizar la documentación de Doxygen.
	\item Explicar el uso del manual y la configuración.
	\end{description}
\end{frame}

\begin{frame}
	En esta ocasión hemos elegido Arch Linux como ambiente de trabajo porque tiene disponible una gran variedad de módulos de DUNE.
	Es recomendable habilitar el repositorio \href{https://wiki.archlinux.org/title/Unofficial\_user\_repositories\#arch4edu}{\texttt{arch4edu}}\footnote{Administrado por Jingbei Li de la Universidad de Tsinghua.}.
\end{frame}

\begin{frame}
	\frametitle{Usando el emulador de terminal}
\end{frame}

\begin{frame}[fragile]\LARGE
\begin{lstlisting}
gitpod ~/dune-basics $ tldr doxygen

doxygen
La documentación del sistema para varios lenguajes se puede encontrar en: 
	http://www.doxygen.nl.

- Para generar un archivo o plantilla de configuración utilice el comando 'Doxyfile':
	doxygen -g

- Para generar un archivo o plantilla de configuración con un nombre específico 'MiDoxygen':
	doxygen -g MiDoxygen

- Genera una plantilla de un archivo de configuración:
	doxygen -g ruta/a/archivo_de_configuracion

- Genera documentación usando un archivo de configuración existente:
	doxygen ruta/a/archivo_de_configuracion
\end{lstlisting}
\end{frame}

\begin{frame}[fragile]\LARGE
En caso de no tener instalado en Arch Linux el programa utilice el comando:
\begin{lstlisting}
gitpod ~/dune-basics $ yay -Sy doxygen --noconfirm
\end{lstlisting}

\begin{lstlisting}
gitpod ~/dune-basics $ doxygen -g MiDoxigen

Configuration file 'MiDoxigen' created.

Now edit the configuration file and enter

  doxygen MiDoxigen

to generate the documentation for your project
\end{lstlisting}
\end{frame}

\begin{frame}
El archivo \lstinline{MiDoxygen} es un archivo de texto plano con
alrededor de $2600$ líneas, que contiene la configuración, su
estructura es del tipo clave valor, con líneas comentadas por \#, a
continuación explicaremos algunas de ellas para hacer una
configuración básicas, en caso de necesitar más información dirijase
a la página~\cite{Doxygen2021}.%\url{https://www.doxygen.nl/manual/index.html}
\end{frame}

\begin{frame}[fragile]\LARGE
\begin{lstlisting}
# El PROJECT_NAME puede ser una palabra simple o una secuencia de palabras 
# entre comillas dobles (a menos que usted utilice Doxywizard) esto es para 
# identificar el proyecto para el que se va a generar la documentación.
# Este nombre es usado en el título de la mayoría de las páginas generadas y en pequeños otros lugares.
# El valor por defecto es: My Project, y se le puede dar el nombre de su proyecto.

PROJECT_NAME           = "Dune-project-cpp-review"
\end{lstlisting}
\end{frame}

\begin{frame}
	Si se utiliza el programa \lstinline{doxywizard} puede configurar los parámetros del archivo \lstinline{doxyfile}.

	\begin{figure}[ht!]
	\centering
	\includegraphics[scale=0.2]{wizard_capture.png}
	\end{figure}

\end{frame}

\begin{frame}[fragile]
	\frametitle{Classes}

	\lstinputlisting[
		caption={Programa \texttt{hello-linux.cc}.},
		label=hello-linux.cc,
	]{../../../sandbox/hello-linux.cc}
\end{frame}

\begin{frame}[fragile]
	\frametitle{Classes}

	\lstinputlisting[
		caption={Programa \texttt{dune-basics.cc}.},
		label=dune-basics.cc,
	]{../../../src/dune-basics.cc}

\end{frame}

\begin{frame}[fragile]
	\frametitle{Classes}

	\lstinputlisting[
		caption={Programa \texttt{dune-basics.cc}.},
		label=dune-basics.cc,
	]{../../../sandbox/dune-math-constants.cc}

\end{frame}

\begin{frame}\transblindsvertical
	\frametitle{Referencias}
	%------------------------------------------------------------ 1
	\only<1>{
		\begin{itemize}
			\item Libros
			      \nocite{*}
			      \printbibliography[heading=none,keyword=book]
		\end{itemize}
	}
	%------------------------------------------------------------ 2
	\only<2>{
		\begin{itemize}
			\item Artículos
			      \printbibliography[heading=none,keyword=paper]
		\end{itemize}
	}
	%------------------------------------------------------------ 3
	\only<3>{
		\begin{itemize}
			\item Sitios web
			      \printbibliography[heading=none,keyword=online]
		\end{itemize}
	}
\end{frame}

\end{document}