\input{dune-cmake-preamble}

\begin{document}

\frame[plain,noframenumbering]{
	\titlepage
}

\begin{frame}[fragile]
	\frametitle{CMake}
	Es un generador de sistemas de compilación de código abierto e
	independiente del compilador y la plataforma, es decir, produce
	instrucciones para otros sistemas de compilación como Makefile,
	Ninja, Visual Studio, Qt Creator, Android Studio y Xcode.
	También incluye características que permiten la instalación,
	empaquetamiento y soporte nativo de pruebas de software.
	
	Objetivos:

	\begin{itemize}
		\item

		Construir diversos ejemplos de proyectos de CMake en los
		lenguajes C/C++/Fortran/Python que construyan ejecutables u
		objetos compartidos/estáticos/interfaces.

		\item

		Correr pruebas con ctest, catch2, gtest, pytest.

		\item

		Usar dependencias de terceros en un proyecto CMake.
	\end{itemize}

	Está orientado a cualquier estudiante, profesor, investigador,
	programador que quiera aprender de Arch Linux y quiera aprender
	a usar efectivamente CMake en un proyecto basado en Dune.

	No se se asume que tenga experiencia en GNU/Linux o C++, pero
	sí familiarizado con algún lenguaje de programación.

	% \lstinline{cmake} es una herramienta para automatizar la
	% generación de \lstinline{Makefile}.

	Tres herramientas de línea de comando

	\begin{description}
		\item[\href{https://man.archlinux.org/man/cmake.1}{\lstinline{/usr/bin/cmake}}]

		\item[\href{https://man.archlinux.org/man/cpack.1}{\lstinline{/usr/bin/cpack}}]

		\item[\href{https://man.archlinux.org/man/ctest.1}{\lstinline{/usr/bin/ctest}}]
	\end{description}

	Tres herramientas interactivas

	\begin{description}
		\item[\href{https://man.archlinux.org/man/ccmake.1}{\lstinline{/usr/bin/ccmake}}]

		\item[\href{https://man.archlinux.org/man/cmake-gui.1}{\lstinline{/usr/bin/cmake-gui}}]
	\end{description}

	% TODO: Teoría
	% Explicar la necesidad de cmake (ver Scott, Rafal)
	% Instalación de cmake en Arch Linux.
	% Mencionar el manual pages de cmake
	% conceptos: build tree, source tree
	% Etapas de cmake con un diagrama.
	% TODO: Práctica
	% Ejemplo: un ejemplo de álgebra lineal con BLAS del libro de Simon.
	% Ejemplo: documentación con doxygen (Berner, Rafal).
	% Ejemplo con ctest: https://coderefinery.github.io/cmake-workshop/testing
\end{frame}

\begin{frame}
	\frametitle{Instalación}
	Esta tecnología imprescindible en el lenguaje C++ se encuentra
	disponible en mayoría de repositorios de distribuciones
	GNU/Linux importantes.
	Si desea tener una versión actual, puede instalar desde el
	repositorio \href{https://archlinux.org/packages/extra/x86_64/cmake}{\lstinline{[extra]}}

	\begin{verbatim}
		[user@host somedir]$ sudo pacman -Syu cmake graphviz plantuml gcovr cppcheck python-cpplint ccache
	\end{verbatim}

\end{frame}

\begin{frame}
	\frametitle{Un archivo \lstinline{CMakeLists.txt} minimal}
	Supongamos que tenemos el siguiente script \lstinline{CMakeLists.txt}

	\inputminted{cmake}{CMakeList.txt.sample}

	Entendamos, este script en más detalle.
	En este script definimos los requisitos para la construcción, desde
	el código fuente y objetivos, pasando por las pruebas, empaquetamiento, etc.
	Y delegaremos la tarea de compilación al programa \lstinline{make}.

	\begin{itemize}
		\item

		\href{https://cmake.org/cmake/help/latest/command/cmake_minimum_required.html}{\lstinline{cmake_minimum_required()}}
		indica la versión mínima de \lstinline{cmake} que requiere para ejecutar.

		\item
		
		\href{https://cmake.org/cmake/help/latest/command/project.html}{\lstinline{project()}}
		define el nombre del proyecto, su número de versión y que
		está escrito en el lenguaje de programación C++.
		
		\item

		\href{https://cmake.org/cmake/help/latest/command/set.html}{\lstinline{set()}}
		asigna la variable de entorno, en este caso establece la versión del estándar C++ 20.

		\item

		\href{https://cmake.org/cmake/help/latest/command/add_executable.html}{\lstinline{add_executable()}}}
		crea un ejecutable a partir de un script en C++.
	\end{itemize}
\end{frame}

\begin{frame}
	\frametitle{El proceso de construcción CMake}
\end{frame}

\begin{frame}
	\frametitle{Tipos de construcción CMake}

	\begin{description}
		\item[Debug]

		\item[Release]

		\item[RelWithDebInfo]

		\item[MinSizeRel]

		\item[None]
	\end{description}
\end{frame}

\begin{frame}
	\frametitle{Generación automática de documentación con CMake}
	La documentación es una parte esencial de cualquier proyecto
	exitoso.

	Veamos cómo integrar Doxygen con CMake para generar
	automáticamante la documentación para proyectos con CMake.

	CMake generará un archivo \lstinline{Doxyfile}.

	Esperamos que Doxygen genere la documentación de la interfaz de
	programación de aplicaciones (API) para cada clase y sus diagramas de herencia con dot de graphviz
\end{frame}

\begin{frame}
	\frametitle{Pruebas con ctest}
\end{frame}

% Reproducible builds, superbuilds

% TODO:
% Se va a separar una diapositiva de dune-cmake cuando esté terminado.
% Ejemplo: que sea de dune con dune-common, dune-geometry cmake.

\begin{frame}
	\frametitle{
		El comando \href{https://gitlab.dune-project.org/core/dune-common/-/raw/master/bin/duneproject}{\lstinline{duneproject}}
	}

	Es un script asistente escrito en el lenguaje \lstinline{bash}
	que se encuentra en \lstinline{duneproject}
	dentro del paquete \lstinline{dune-common}.
\end{frame}

% \begin{frame}\transblindsvertical
% 	\frametitle{Referencias}
% 	%------------------------------------------------------------ 1
% 	\only<1>{
% 		\begin{itemize}
% 			\item Libros
% 			      \nocite{*}
% 			      \printbibliography[heading=none,keyword=book]
% 		\end{itemize}
% 	}
% 	%------------------------------------------------------------ 2
% 	\only<2>{
% 		\begin{itemize}
% 			\item Artículos
% 			      \printbibliography[heading=none,keyword=paper]
% 		\end{itemize}
% 	}
% 	%------------------------------------------------------------ 3
% 	\only<3>{
% 		\begin{itemize}
% 			\item Sitios web
% 			      \printbibliography[heading=none,keyword=online]
% 		\end{itemize}
% 	}
% \end{frame}

\end{document}
%https://cliutils.gitlab.io/modern-cmake
%https://www.dune-project.org/sphinx/content/sphinx/core/index.html
%https://www.dune-project.org/sphinx/content/sphinx/core-2.7
%https://github.com/toeb/moderncmake/blob/master/Modern%20CMake.pdf