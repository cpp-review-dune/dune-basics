\input{dune-cmake-preamble}

\begin{document}

\frame[plain,noframenumbering]{
	\titlepage
}

\begin{frame}[fragile]
	Supongamos que tenemos el siguiente script \mintinline{plain}{CMakeLists.txt}

	\inputminted{cmake}{CMakeList.txt.sample}

	En este script definimos los requisitos para la construcción, desde
	el código fuente y objetivos, pasando por las pruebas, empaquetamiento, etc.
	Y delegaremos la tarea de compilación al programa \mintinline{bash}{make}.

	\begin{itemize}
		\item

		\href{https://cmake.org/cmake/help/latest/command/cmake_minimum_required.html}{
			\mintinline{cmake}{cmake_minimum_required()}
		} indica la versión mínima de \mintinline{bash}{cmake} que requiere para ejecutar.

		\item
		
		\href{https://cmake.org/cmake/help/latest/command/project.html}{\mintinline{cmake}{project()}}
		define el nombre del proyecto, su número de versión y que
		está escrito en el lenguaje de programación C++.
		
		\item

		\href{https://cmake.org/cmake/help/latest/command/set.html}{\mintinline{cmake}{set()}}
		asigna la variable de entorno, en este caso establece la versión del estándar C++ 20.

		\item

		\href{https://cmake.org/cmake/help/latest/command/add_executable.html}{\mintinline{cmake}{add_executable()}}
		crea un ejecutable a partir de un script en C++.
	\end{itemize}
\end{frame}

\begin{frame}
	\frametitle{
		El comando \href{https://gitlab.dune-project.org/core/dune-common/-/raw/master/bin/duneproject}{\mintinline{cmake}{duneproject}}
	}

	Es un script asistente escrito en el lenguaje \mintinline{bash}{bash}
	que se encuentra en \mintinline{bash}{/usr/bin/duneproject}
	dentro del paquete \lstinline{dune-common}.
\end{frame}

\begin{frame}\transblindsvertical
	\frametitle{Referencias}
	%------------------------------------------------------------ 1
	\only<1>{
		\begin{itemize}
			\item Libros
			      \nocite{*}
			      \printbibliography[heading=none,keyword=book]
		\end{itemize}
	}
	%------------------------------------------------------------ 2
	\only<2>{
		\begin{itemize}
			\item Artículos
			      \printbibliography[heading=none,keyword=paper]
		\end{itemize}
	}
	%------------------------------------------------------------ 3
	\only<3>{
		\begin{itemize}
			\item Sitios web
			      \printbibliography[heading=none,keyword=online]
		\end{itemize}
	}
\end{frame}

\end{document}
%https://cliutils.gitlab.io/modern-cmake
%https://www.dune-project.org/sphinx/content/sphinx/core/index.html
%https://sci-hub.ru/10.1088/1742-6596/396/5/052021
%https://sci-hub.ru/10.1201/9781315382395-9
%https://discourse.cmake.org/t/professional-cmake-a-practical-guide-12th-edition/5440
%https://www.packtpub.com/product/modern-cmake-for-c/9781801070058
%https://github.com/PacktPublishing/CMake-Best-Practices