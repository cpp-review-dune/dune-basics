\input{dune-cmake-preamble}

\begin{document}

\frame[plain,noframenumbering]{
	\titlepage
}

\begin{frame}[fragile]
	\lstinline{cmake} es una herramienta para automatizar la
	generación de \lstinline{Makefile}.
	Supongamos que tenemos el siguiente script \lstinline{CMakeLists.txt}

	\inputminted{cmake}{CMakeList.txt.sample}

	En este script definimos los requisitos para la construcción, desde
	el código fuente y objetivos, pasando por las pruebas, empaquetamiento, etc.
	Y delegaremos la tarea de compilación al programa \lstinline{make}.

	\begin{itemize}
		\item

		\href{https://cmake.org/cmake/help/latest/command/cmake_minimum_required.html}{
			\lstinline{cmake_minimum_required()}}
		} indica la versión mínima de \lstinline{cmake} que requiere para ejecutar.

		\item
		
		\href{https://cmake.org/cmake/help/latest/command/project.html}{\lstinline{project()}}
		define el nombre del proyecto, su número de versión y que
		está escrito en el lenguaje de programación C++.
		
		\item

		\href{https://cmake.org/cmake/help/latest/command/set.html}{\lstinline{set()}}
		asigna la variable de entorno, en este caso establece la versión del estándar C++ 20.

		\item

		\href{https://cmake.org/cmake/help/latest/command/add_executable.html}{\lstinline{add_executable()}}}
		crea un ejecutable a partir de un script en C++.
	\end{itemize}
\end{frame}

\begin{frame}
	\frametitle{
		El comando \href{https://gitlab.dune-project.org/core/dune-common/-/raw/master/bin/duneproject}{\lstinline{duneproject}}
	}

	Es un script asistente escrito en el lenguaje \lstinline{bash}
	que se encuentra en \lstinline{duneproject}
	dentro del paquete \lstinline{dune-common}.
\end{frame}

\begin{frame}\transblindsvertical
	\frametitle{Referencias}
	%------------------------------------------------------------ 1
	\only<1>{
		\begin{itemize}
			\item Libros
			      \nocite{*}
			      \printbibliography[heading=none,keyword=book]
		\end{itemize}
	}
	%------------------------------------------------------------ 2
	\only<2>{
		\begin{itemize}
			\item Artículos
			      \printbibliography[heading=none,keyword=paper]
		\end{itemize}
	}
	%------------------------------------------------------------ 3
	\only<3>{
		\begin{itemize}
			\item Sitios web
			      \printbibliography[heading=none,keyword=online]
		\end{itemize}
	}
\end{frame}

\end{document}
%https://cliutils.gitlab.io/modern-cmake
%https://www.dune-project.org/sphinx/content/sphinx/core/index.html
%https://iopscience.iop.org/article/10.1088/1742-6596/396/5/052021