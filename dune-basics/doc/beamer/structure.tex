% arara: clean: {
% arara: --> extensions:
% arara: --> ['log','pdf','aux']
% arara: --> }
% arara: lualatex: {
% arara: --> shell: yes,
% arara: --> interaction: batchmode
% arara: --> }
% arara: clean: {
% arara: --> extensions:
% arara: --> ['log','aux']
% arara: --> }
\documentclass[9pt,a3paper]{scrartcl}
\usepackage[margin=.75in]{geometry}
\usepackage[spanish,es-sloppy]{babel}%english,
\usepackage{dirtree}
\usepackage{wrapfig}
\usepackage{hyperref}
\pagestyle{empty}

\parskip1ex
\setlength{\parindent}{0ex}

\begin{document}

\par\bigskip
\begin{center}\sffamily\bfseries
	\Huge{
		Estructura de directorios en
		\href{https://gitlab.dune-project.org/pdelab/dune-pdelab-tutorials}{\texttt{dune-pdelab-tutorials}}
	} \par
	\LARGE{
		\href{https://github.com/cpp-review-dune}{C++ Review Dune}
	}
\end{center}
\rule{\textwidth}{1pt}
\bigskip

\begin{wrapfigure}{l}{5.5cm}
	\dirtree{%
		.1 tutorial-<usuario>.
		.2 BSD.
		.2 c++.
		.3 CMakeLists.txt.
		.3 doc.
		.3 exercise.
		.3 slides.
		.2 CC-BY-SA.
		.2 cmake.
		.3 modules.
		.2 CMakeLists.txt.
		.2 config.h.cmake.
		.2 COPYING.
		.2 dune.module.
		.2 dune-pdelab-tutorials.pc.in.
		.2 gridinterface.
		.3 CMakeLists.txt.
		.3 exercise.
		.3 slides.
		.2 latexstyle.
		.3 exercise.sty.
		.2 LICENSE.md.
		.2 overview.
		.3 abstractions.bib.
		.3 abstractions.tex.
		.3 CMakeLists.txt.
		.3 debug.opts.
		.3 exercise.sty.
		.3 exercise-workflow.tex.
		.3 overview.tex.
		.3 release.opts.
		.2 README.md.
		.2 stamp-vc.
		.2 tutorial00.
		.3 CMakeLists.txt.
		.3 doc.
		.3 exercise.
		.3 slides.
		.3 src.
		.2 tutorial01.
		.3 CMakeLists.txt.
		.3 doc.
		.3 exercise.
		.3 slides.
		.3 src.
		.2 tutorial02.
		.3 CMakeLists.txt.
		.3 doc.
		.3 slides.
		.3 src.
		.2 tutorial03.
		.3 CMakeLists.txt.
		.3 doc.
		.3 exercise.
		.3 slides.
		.3 src.
		.2 tutorial04.
		.3 CMakeLists.txt.
		.3 doc.
		.3 exercise.
		.3 slides.
		.3 src.
		.2 tutorial05.
		.3 CMakeLists.txt.
		.3 doc.
		.3 exercise.
		.3 slides.
		.3 src.
		.2 tutorial06.
		.3 CMakeLists.txt.
		.3 doc.
		.3 exercise.
		.3 slides.
		.3 src.
		.2 tutorial07.
		.3 CMakeLists.txt.
		.3 doc.
		.3 exercise.
		.3 slides.
		.3 src.
		.2 tutorial08.
		.3 CMakeLists.txt.
		.3 doc.
		.3 src.
		.2 tutorial09.
		.3 CMakeLists.txt.
		.3 exercise.
		.3 slides.
		.3 src.
		.2 TUTORIALLIST.
		.2 workflow.
		.3 CMakeLists.txt.
		.3 exercise.
		.3 slides.
	}
\end{wrapfigure}

\Large

Los programas del repositorio
\verb!\href{https://gitlab.dune-project.org/pdelab/dune-pdelab-tutorials}{dune-pdelab-tutorials}!
se ejecutarán en \href{https://gitpod.io}{Gitpod}, una aplicación de
\href{https://kubernetes.io}{Kubernetes} de código abierto para
entornos de desarrollo automatizados y listos para utilizar fuera de
la caja, que nos otorga $50$ horas gratuitas de uso durante cada mes.
El \href{https://www.redhat.com/es/topics/virtualization/what-is-KVM}{hipervisor KVM}
que emplea emula un sistema operativo basado en
\href{http://archlinux.org}{Arch Linux} con el módulo
\href{https://aur.archlinux.org/packages/dune-pdelab}{\texttt{dune-pdelab}}
más \href{https://raw.githubusercontent.com/cpp-review-dune/introductory-review/main/src/Docker/DUNEPDELabTutorials.Dockerfile}{algunas herramientas}
instaladas.
En este entorno, tendrá un directorio
\verb!/workspace/tutorial-<usuario>!, que tiene la estructura
mostrada a la izquierda.
Una vista de árbol de este se muestra en el panel izquierdo, la lista
completa de ejercicios está en el archivo
\href{https://gitlab.dune-project.org/pdelab/dune-pdelab-tutorials/-/raw/master/TUTORIALLIST}{\texttt{TUTORIALLIST}}.

\

DUNE Numerics utiliza \verb!cmake! como sistema de construcción.
En \verb!cmake!, hay una clara separación entre el
\textit{directorio fuente}, que generalmente está versionado
(aquí: el subdirectorio \verb!tutorial-<usuario>!) y el
\textit{directorio de compilación}, donde se encuentran los
ejecutables, documentación generado por Lua\LaTeX{}, archivos
\href {https://vtk.org}{\texttt{vtk}},
\href{http://www.gnuplot.info}{\texttt{gnuplot}}, etc.
El directorio de compilación será \verb!/workspace/build!.

\

El directorio \verb!/workspace/build! refleja la estructura del
directorio fuente.
Debería navegar por defecto al directorio ``build'' del ejercicio
actual y trabajar allí.
Cada directorio de este módulo corresponde a un tutorial
(\verb!tutorial0[0-9]!).
Todos comparten la siguiente estructura: el directorio \verb!src!
contiene el código de ejemplo que se mostró en la conferencia.
El directorio \verb!doc! contiene las fuentes de \LaTeX{} de una
explicación detallada del tutorial.
El directorio \verb!exercise!, que es relevante para este curso, se
subdivide aún más: \verb!task! contiene el código esqueleto sobre el
que trabajar durante el ejercicio, \verb!doc! contiene la fuentes de
la hoja de ejercicios y \verb!solution! contiene lo que esperas
obtener.

\

En caso de que no esté familiarizado con un sistema GNU/Linux, debajo
encontrará una pequeña lista de comandos frecuentes para el
desarrollo de los ejercicios:
\begin{itemize}
	\item

	      \verb!sudo pacman -Syyu --noconfirm! actualiza la base de
	      datos de los repositorios e instala paquetes actualizados.

	\item

	      \verb!man <command>! muestra las páginas del manual del
	      comando, en caso de estar disponible, digite \verb!q! para
	      salir.
	      También es válido \verb!<command> --help!.

	\item

	      \verb!tldr <command>! muestra las páginas del manual de la
	      comunidad.

	\item

	      \verb!mkdir -p <name>! crea una carpeta \verb!<name>! sin
	      contenido en el espacio de trabajo actual.

	\item

	      \verb!cmake -S /workspace/tutorial-<usuario> -B /workspace/build!
	      crea una carpeta
	      \verb!build! sin contenido en el espacio de trabajo actual.
	      En general, \verb!cmake -S ruta/al/directorio/padre/a/CMakeLists.txt!.

	\item

	      \verb!cd <dir>! cambia el directorio de trabajo actual
	      a \verb!dir! (y al directorio del usuario, si se omite la ruta).

	\item

	      \verb!ls! lista el contenido del directorio de trabajo actual.

	\item

	      \verb!pwd! imprime el directorio de trabajo.

	\item

	      \verb!g++ <options> <sources>! compila fuentes de C++
	      (por ejemplo en el primer ejercicio de C++).

	\item

	      \verb!make <executablename>! (re)compila ejecutables en el
	      directorio de compilación actual.
	      Si se omite el nombre del ejecutable, se compilan todos los
	      ejecutables del directorio actual.

	\item

	      \verb!paraview! es un programa de visualización para archivos
	      de extensión
	      \verb!vtk! o \verb!vtu!.
\end{itemize}

\section*{\Huge Recursos}

\begin{description}
	\item[Sección de tutoriales]
		\leavevmode

		\begin{itemize}
			\item

			      \url{https://wiki.archlinux.org/title/Bash}

			\item

			      \url{https://wiki.archlinux.org/title/Classroom}

			\item

			      \url{https://wiki.archlinux.org/title/CMake_package_guidelines}
		\end{itemize}


	\item[Visores DICOM y renderizado de volumen]
		\leavevmode
		\begin{itemize}
			\item

			      \url{https://wiki.archlinux.org/title/List_of_applications/Science#DICOM_viewers_and_volume_rendering}
		\end{itemize}

	\item[C++ Review dune]
		\leavevmode
		\begin{itemize}
			\item

			      \url{https://cpp-review-dune.github.io}
		\end{itemize}


	\item[\href{https://github.com/precice/tutorials/pull/264/files}{Créditos a preCICE\#264}]
		\leavevmode
		\begin{itemize}
			\item

			      \url{https://merely-useful.tech/py-rse}

			\item

			      \url{https://missing.csail.mit.edu}

			\item

			      \url{https://simulation-software-engineering.github.io/homepage}
		\end{itemize}

\end{description}

\section*{\Huge Pasos en Gitpod}

\begin{enumerate}
	\item

	      Inicie sesión en \href{https://github.com/login}{GitHub} y en
	      \href{https://gitpod.io/login}{Gitpod}.
	      Ajuste los permisos de escritura en \url{https://gitpod.io/integrations}.

	\item

	      Acepte la invitación del GitHub Classroom
	      \url{https://classroom.github.com/a/uciSBZ_m}.

	\item

	      Ingrese a la liga del repositorio generado
	      \url{https://github.com/cpp-review-dune/tutorial-<usuario>}.

	\item

	      Acceda a
	      \url{gitpod.io/#https://github.com/cpp-review-dune/tutorial-<usuario>}.
\end{enumerate}

\vfill

\footnotemark{
	Adaptación de
	\href{https://dune-pdelab-course.readthedocs.io/en/latest/intro.html#how-to-study-with-the-material}{
		\emph{Structure of the course material for the IWR Dune Course, March $2021$}
	}.
}
%\selectlanguage{spanish}
%\selectlanguage{english}
\end{document}