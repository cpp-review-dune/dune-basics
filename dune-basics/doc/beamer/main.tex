\documentclass[
	spanish,
	10pt,
	xcolor=table,
	handout,
	aspectratio=1610,
  ignorenonframetext
]{beamer}

\usepackage[spanish]{babel}
\usepackage[
  cursointroductorio,
  fullpagenumbering
]{dunestyle-beamer}
\usepackage{booktabs}
\usepackage{longtable}

\title{
  Introducción al módulo \texttt{dune-common}
}
\subtitle{
  Conceptos básicos:
  \lstinline{bash},
  \lstinline{g++},
  \lstinline{cmake},
  \lstinline{gnuplot},
  \lstinline{gmsh},
  \lstinline{paraview}
}

\author{}
\institute[]
{
  
  \begin{center}
    Practique en GitPod
    \href{https://gitpod.io\#/https://github.com/cpp-review-dune/dune-basics}{\includegraphics[width=4.5cm]{open-in-gitpod}}
	\end{center}

  \begin{center}
  Disponible en \href{https://github.com/cpp-review-dune}{\includegraphics[width=0.5cm]{GitHub-Mark-120px-plus}}
  \end{center}

  \begin{center}
  ¡Únete al grupo en Telegram!
  \href{https://t.me/joinchat/OsfYP1xnFlxjN2Ix}{\includegraphics[width=0.5cm]{telegram-logo}}
  \end{center}
}

\begin{document}

\frame[plain,noframenumbering]{\titlepage}

\begin{frame}
  \frametitle{Objetivos de esta introdución}
  A number of advanced.
  \begin{itemize}
    \item Correr \lstinline{docker} en sistemas tipo Unix.
  \end{itemize}
\end{frame}

\begin{frame}
  \frametitle{Conociendo el sistema operativo \href{https://archlinux.org}{Arch Linux}}

  Es una distribución GNU/Linux de propósito general, desarrollada independientemente para procesadores x86-64, que se adhiere a los principios de simplicidad, modernidad, pragmatismo, centrado en usuarios y versatilidad.
  % \footnote{El líder del proyecto es \href{https://wiki.archlinux.org/title/User:Anthraxx}{\texttt{Anthraxx}}, desarrollador alemán del kernel \href{https://github.com/anthraxx/linux-hardened}{\texttt{linux-hardened}}.}

  \

  Proporciona las últimas versiones estables de la mayoría del software siguiendo el modelo de \href{https://en.wikipedia.org/wiki/Rolling\_release}{lanzamiento continuo}, no existen versiones como en Ubuntu 20.04, 21.10, 22.04, etc.

  \

  Arch está respaldado por \lstinline{pacman}, un gestor de paquetes ligero, sencillo y rápido, que permite actualizar todo el sistema con una orden.
  Los scripts \href{https://wiki.archlinux.org/title/PKGBUILD}{PKGBUILD} aportados por la comunidad para la elaboración desde las fuentes, como los módulos de DUNE, se encuentran en el \href{http://aur.archlinux.org}{\emph{Arch User Repository}}.

  \begin{table}[ht!]
    \caption{Comparación de la línea de comando del gestión de software (fuente: \url{wiki.archlinux.org})}
    \centering\footnotesize
    \begin{tabular}{cccp{50pt}cc}
      \toprule
      Acción             & Arch                    & Red Hat/Fedora          & Debian/Ubuntu                               & SLES/openSUSE           & Gentoo
      \tabularnewline
      \midrule
      Instala paquetes   & \lstinline|pacman -S|  & \lstinline|dnf install|  & \lstinline|apt install|                      & \lstinline|zypper install|  & \lstinline|emerge -a|
      \tabularnewline
      Elimina paquetes   & \lstinline|pacman -Rs|  & \lstinline|dnf remove|  & \lstinline|apt remove|                      & \lstinline|zypper remove|  & \lstinline|emerge -C|
      \tabularnewline
      Busca paquetes     & \lstinline|pacman -Ss| & \lstinline|dnf search| & \lstinline|apt search|                     & \lstinline|zypper search| & \lstinline|emerge -S|
      \tabularnewline
      Actualiza paquetes & \lstinline|pacman -Syu| & \lstinline|dnf upgrade| & \lstinline|apt update \&\&|\newline apt upgrade & \lstinline|zypper update| & \lstinline|emerge -u world|
      \tabularnewline
      \bottomrule
    \end{tabular}
  \end{table}

\end{frame}

\begin{frame}
  \frametitle{Filosofía de Arch}

  \begin{description}
    \item[Simplicidad]
          Es minimalista.
          Se usa herramienta pequeñas que sigue la filosofía de UNIX, de modo que tengas una base muy pequeña que deje configurar la máquina de manera más cómoda para el usuario.
    \item[Modernidad]
          Mantener la paquetería lo más actualizada posible sin sacrificar la estabilidad. Cuenta con los compiladores más actuales de \lstinline{gcc}, \lstinline{lualatex}, \lstinline{go}, etc.
    \item[Pragmatismo]
          La gran cantidad de paquetes y scripts de compilación en los diversos repositorios de Arch Linux ofrecen software gratuito y de código abierto para quienes lo prefieren, así como paquetes de software propietario para quienes adoptan la funcionalidad por encima de la ideología.
    \item[Centrado a las usuarias y usuarios]
          La distribución está destinada a satisfacer las necesidades de quienes contribuyen a ella, en lugar de intentar atraer a tantos usuarios como sea posible.
          Está dirigido al usuario competente de GNU/Linux, o cualquier persona con una actitud de hágalo usted mismo que esté dispuesto a leer la documentación y resolver sus propios problemas.
          Se anima a todos los usuarios a participar y contribuir a la distribución.
          Informar y ayudar a corregir errores es muy valioso y los parches que mejoran los paquetes o los proyectos principales son muy apreciados: los desarrolladores de Arch son voluntarios y los contribuyentes activos a menudo se convertirán en parte de ese equipo.
    \item[Versatilidad]
  \end{description}

\end{frame}

\begin{frame}

  En esta ocasión hemos elegido Arch Linux como ambiente de trabajo porque tiene disponible una gran variedad de módulos de DUNE.
  Es recomendable habilitar el repositorio \href{https://wiki.archlinux.org/title/Unofficial\_user\_repositories\#arch4edu}{\texttt{arch4edu}}\footnote{Administrado por Jingbei Li de la Universidad de Tsinghua.}.
\end{frame}

\end{document}